\documentclass[12pt]{article}

\usepackage{import}
\import{../../doc}{includes}

\begin{document}
	
	\import{../../doc}{course-project_title}
	
	\subsection*{Условие}
	
	Реализовать программы клиента и сервера, в которых клиент отправляет запросы на сервер, а сервер обрабатывает эти запросы, клиент и сервер связаны очередью сообщений.	
	
	\subsection*{Метод решения}
	
	У клиента есть несколько опций: (регистрируется он в обязательном порядке) создать комнату, присоединиться к комнате, вывести статистику игрока, начать игру. 
	
	\subsection*{Код программы}
	
	\subsubsection*{zmq-utils.hpp}
	
	\lstinputlisting[language=C]{../include/zmq_utils.hpp}
	
	\subsubsection*{server-utils.hpp}
	
	\lstinputlisting[language=C]{../include/server_utils.hpp}
	
	\subsubsection*{utils.hpp}
	
	\lstinputlisting[language=C]{../include/utils.hpp}
	
	\subsubsection*{zmq-utils.cpp}
	
	\lstinputlisting[language=C]{../src/zmq_utils.cpp}
	
	\subsubsection*{client.cpp}
	
	\lstinputlisting[language=C]{../src/client.cpp}
	
	\subsubsection*{server.cpp}
	
	\lstinputlisting[language=C]{../src/server.cpp}
	
	\subsection*{Выводы}
	
	В процессе выполнения лабораторной работы были приобретены навыки практического применения и создания классов и <<статиков>>, также были расширены знания в области использования сокетов очередей сообщений.
	
\end{document}

