\documentclass[12pt]{article}

\usepackage[utf8x]{inputenc}
\usepackage[T1, T2A]{fontenc}
\usepackage{fullpage}
\usepackage{multicol,multirow}
\usepackage{tabularx}
\usepackage{ulem}
\usepackage{listings} 
\usepackage[english,russian]{babel}
\usepackage{tikz}
\usepackage{pgfplots}
\usepackage{indentfirst}
\usepackage{ulem} 


\parindent=1cm
\makeatletter
\newcommand{\rindex}[2][\imki@jobname]{%
	\index[#1]{\detokenize{#2}}%
}
\makeatother
\newcolumntype{P}[1]{>{\raggedbottom\arraybackslash}p{#1}}

\linespread{1}
\pgfplotsset{compat=1.16}
\begin{document}
	
	\section*{\centering Курсовой проект по курсу :\\ Операционные системы}
	
	Выполнил студент группы М8О-201Б-21 \,\, Кварацхелия Александр.
	
	\subsection*{Условие}
	
	Реализовать программы клиента и сервера, в которых клиент отправляет запросы на сервер, а сервер обрабатывает эти запросы, клиент и сервер связаны очередью сообщений.	
	
	\subsection*{Метод решения}
	
	У клиента есть несколько опций: (регистрируется он в обязательном порядке) создать комнату, присоединиться к комнате, вывести статистику игрока, начать игру. 
	
	\subsection*{Код программы}
	
	\subsubsection*{zmq-utils.hpp}
	
	\lstinputlisting[language=C]{../include/zmq_utils.hpp}
	
	\subsubsection*{server-utils.hpp}
	
	\lstinputlisting[language=C]{../include/server_utils.hpp}
	
	\subsubsection*{utils.hpp}
	
	\lstinputlisting[language=C]{../include/utils.hpp}
	
	\subsubsection*{zmq-utils.cpp}
	
	\lstinputlisting[language=C]{../src/zmq_utils.cpp}
	
	\subsubsection*{client.cpp}
	
	\lstinputlisting[language=C]{../src/client.cpp}
	
	\subsubsection*{server.cpp}
	
	\lstinputlisting[language=C]{../src/server.cpp}
	
	\subsection*{Выводы}
	
	В процессе выполнения лабораторной работы были приобретены навыки практического применения и создания классов и <<статиков>>, также были расширены знания в области использования сокетов очередей сообщений.
	
\end{document}

