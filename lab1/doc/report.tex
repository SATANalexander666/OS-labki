\documentclass[12pt]{article}

\usepackage{import}
\import{../../doc}{includes}

\begin{document}
	
	\import{../../doc}{lab1_title}
	
	\subsection*{Условие}
	
	Ознакомиться с сигналами операционной системы UNIX/LINUX, используя утилиту strace, проанализировать результаты, сопоставить их с кодом программы.

	\subsection*{Метод решения}
		
	Использовать свободно распространяемую утилиту strace следующим образом: \\
	\lstinline[]|strace lab2|
	
	\subsection*{Вывод  strace}
	
	\lstinputlisting{../src/strace_output.txt}
	
	\subsection*{Выводы}
	
	Вызов $fork$ дублирует породивший его процесс со всеми его переменными, файловыми дескрипторами, приоритетами процесса, рабочий и корневой каталоги, и сегментами выделенной памяти.
	
	Ребёнок {\bf не} наследует:
	\begin{itemize}
		\item идентификатора процесса (PID, PPID);
		\item израсходованного времени ЦП (оно обнуляется);
		\item сигналов процесса-родителя, требующих ответа;
		\item блокированных файлов (record locking).
	\end{itemize}
	
	В процессе выполнения лабораторной работы были приобретены навыки практического применения создания, обработки и отслеживания их состояния. Для выполнения данного варианта задания создание потоков как таковых не требуется, так как всю работу выполняет системный вызов <<exec>>.
	
\end{document}

