\documentclass[12pt]{article}

\usepackage[utf8x]{inputenc}
\usepackage[T1, T2A]{fontenc}
\usepackage{fullpage}
\usepackage{multicol,multirow}
\usepackage{tabularx}
\usepackage{ulem}
\usepackage{listings} 
\usepackage[english,russian]{babel}
\usepackage{tikz}
\usepackage{pgfplots}
\usepackage{indentfirst}
\usepackage{ulem} 


\parindent=1cm
\makeatletter
\newcommand{\rindex}[2][\imki@jobname]{%
	\index[#1]{\detokenize{#2}}%
}
\makeatother
\newcolumntype{P}[1]{>{\raggedbottom\arraybackslash}p{#1}}

\linespread{1}
\pgfplotsset{compat=1.16}
\begin{document}
	
	\section*{\centering Лабораторная работа №\,2 по курсу :\\ Операционные системы}
	
	Выполнил студент группы М8О-201Б-21 \,\, Кварацхелия Александр.
	
	\subsection*{Условие}
	
	Составить и отладить программу на языке Си, осуществляющую работу с процессами и
	взаимодействие между ними в операционной системе UNIX/LINUX. В результате работы
	программа (основной процесс) должен создать для решение задачи один или несколько
	дочерних процессов. Взаимодействие между процессами осуществляется через системные
	сигналы/события и/или каналы ($pipe$).
	Необходимо обрабатывать системные ошибки, которые могут возникнуть в результате работы.
	
	\subsection*{Задание}
	
	Родительский процесс создает два дочерних процесса. Первой строкой пользователь в консоль
	родительского процесса вводит имя файла, которое будет использовано для открытия File с таким
	именем на запись для child1. Аналогично для второй строки и процесса child2. Родительский и
	дочерний процесс должны быть представлены разными программами.
	
	\par 
	
	Родительский процесс принимает от пользователя строки произвольной длины и пересылает их в
	$pipe1$ или в $pipe2$ в зависимости от правила фильтрации. Процесс $child1$ и $child2$ производят работу
	над строками. Процессы пишут результаты своей работы в стандартный вывод.
	
	\par 
	
	Правило фильтрации: с вероятностью 80\% строки отправляются в pipe1, иначе в pipe2.
	Дочерние процессы удаляют все гласные из строк.
	
	\subsection*{Код программы}
	
	\subsubsection*{utils.h}
	
	\lstinputlisting[language=C]{../include/utils.h}
	
	\subsubsection*{parent.h}
	
	\lstinputlisting[language=C]{../include/parent.h}
	
	\subsubsection{utils.c}
	
	\lstinputlisting[language=C]{../src/utils.c}
	
	\subsubsection{parent.c}
	
	\lstinputlisting[language=C]{../src/parent.c}
	
	\subsubsection{child.c}
	
	\lstinputlisting[language=C]{../src/child.c}
	
	\subsection*{Выводы}
	
	Вызов $fork$ дублирует породивший его процесс со всеми его переменными, файловыми дескрипторами, приоритетами процесса, рабочий и корневой каталоги, и сегментами выделенной памяти.
	
	Ребёнок {\bf не} наследует:
	\begin{itemize}
		\item идентификатора процесса (PID, PPID);
		\item израсходованного времени ЦП (оно обнуляется);
		\item сигналов процесса-родителя, требующих ответа;
		\item блокированных файлов (record locking).
	\end{itemize}
	
	В процессе выполнения лабораторной работы были приобретены навыки практического применения создания, обработки и отслеживания их состояния. Для выполнения данного варианта задания создание потоков как таковых не требуется, так как всю работу выполняет системный вызов <<exec>>.
	
\end{document}

