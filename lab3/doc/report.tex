\documentclass[12pt]{article}

\usepackage{import}
\import{../../doc}{includes}

\begin{document}
	
	\import{../../doc}{lab3_title}
	
	\subsection*{Условие}
	
	Составить программу на языке Си, обрабатывающую данные в многопоточном режиме. При 
	обработки использовать стандартные средства создания потоков операционной системы 
	(Windows/Unix). Ограничение потоков должно быть задано ключом запуска вашей программы.
	Так же необходимо уметь продемонстрировать количество потоков, используемое вашей 
	программой с помощью стандартных средств операционной системы.
	В отчете привести исследование зависимости ускорения и эффективности алгоритма от входящих 
	данных и количества потоков. Получившиеся результаты необходимо объяснить.
	
	\subsection*{Задание}
	
	Отсортировать массив целых чисел при помощи TimSort.
	
	\subsection*{Метод решения}
	
	Составленный алгоритм соответсвует принципу $completely lockless$, заключающийся в том, что каждый поток изменяет только те данные, которые не изменяют другие потоки.
	
	\subsection*{Код программы}
	
	\subsubsection*{main.cpp}
	
	\lstinputlisting[language=C++]{../main.cpp}
	
	\subsubsection*{utils.hpp}
	
	\lstinputlisting[language=C++]{../include/utils.hpp}
	
	\subsubsection*{body.hpp}
	
	\lstinputlisting[language=C++]{../include/body.hpp}
	
	\subsubsection*{utils.cpp}
	
	\lstinputlisting[language=C++]{../src/utils.cpp}
		
	\subsection*{Выводы}
	
	В процессе выполнения лабораторной работы были приобретены навыки практического применения создания, обработки и отслеживания состояния потоков. Для выполнения данного варианта задания использование примитивов синхронизации не требуется.
	
\end{document}

