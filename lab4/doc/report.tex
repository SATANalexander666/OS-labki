\documentclass[12pt]{article}

\usepackage{import}
\import{../../doc}{includes}

\begin{document}
	
	\import{../../doc}{lab4_title}
	
	\subsection*{Условие}
	
	Составить и отладить программу на языке Си, осуществляющую работу с процессами и
	взаимодействие между ними в операционной системе UNIX/LINUX. В результате работы
	программа (основной процесс) должен создать для решение задачи один или несколько
	дочерних процессов. Взаимодействие между процессами осуществляется через системные
	сигналы/события и/или каналы ($pipe$).
	Необходимо обрабатывать системные ошибки, которые могут возникнуть в результате работы.
	
	\subsection*{Задание}
	
	Родительский процесс создает два дочерних процесса. Первой строкой пользователь в консоль
	родительского процесса вводит имя файла, которое будет использовано для открытия File с таким
	именем на запись для child1. Аналогично для второй строки и процесса child2. Родительский и
	дочерний процесс должны быть представлены разными программами.
	
	\par 
	
	Родительский процесс принимает от пользователя строки произвольной длины и пересылает их в
	$pipe1$ или в $pipe2$ в зависимости от правила фильтрации. Процесс $child1$ и $child2$ производят работу
	над строками. Процессы пишут результаты своей работы в стандартный вывод.
	
	\par 
	
	Правило фильтрации: с вероятностью 80\% строки отправляются в pipe1, иначе в pipe2.
	Дочерние процессы удаляют все гласные из строк.
	
	\subsection*{Код программы}
	
	\subsubsection*{utils.h}
	
	\lstinputlisting[language=C]{../include/utils.h}
	
	\subsubsection*{parent.h}
	
	\lstinputlisting[language=C]{../include/parent.h}
	
	\subsubsection{utils.c}
	
	\lstinputlisting[language=C]{../src/utils.c}
	
	\subsubsection*{parent.c}
	
	\lstinputlisting[language=C]{../src/parent.c}
	
	\subsection*{Выводы}
	
	В процессе выполнения лабораторной работы были приобретены навыки реализации обмена информацией с помощью файлов, находящихся в общей памяти процессов.
	
\end{document}

