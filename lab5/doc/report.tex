\documentclass[12pt]{article}

\usepackage[utf8x]{inputenc}
\usepackage[T1, T2A]{fontenc}
\usepackage{fullpage}
\usepackage{multicol,multirow}
\usepackage{tabularx}
\usepackage{ulem}
\usepackage{listings} 
\usepackage[english,russian]{babel}
\usepackage{tikz}
\usepackage{pgfplots}
\usepackage{indentfirst}
\usepackage{ulem} 


\parindent=1cm
\makeatletter
\newcommand{\rindex}[2][\imki@jobname]{%
	\index[#1]{\detokenize{#2}}%
}
\makeatother
\newcolumntype{P}[1]{>{\raggedbottom\arraybackslash}p{#1}}

\linespread{1}
\pgfplotsset{compat=1.16}
\begin{document}
	
	\section*{\centering Лабораторная работа №\,5 по курсу :\\ Операционные системы}
	
	Выполнил студент группы М8О-201Б-21 \,\, Кварацхелия Александр.
	
	\subsection*{Условие}
	
	Ознакомиться с сигналами операционной системы UNIX/LINUX, используя утилиту strace, проанализировать результаты, сопоставить их с кодом программы.
	
	
	\subsection*{Метод решения}
	
	Использовать свободно распространяемую утилиту strace следующим образом: \\
	\lstinline[]|strace lab2|
	
	\subsection*{Код программы}
	
	\subsubsection*{libLab5.h}
	
	\lstinputlisting[language=C]{../include/libLab5.h}
	
	\subsubsection*{lib1.c}
	
	\lstinputlisting[language=C]{../src/lib1.c}
	
	\subsubsection*{lib2.c}
	
	\lstinputlisting[language=C]{../src/lib2.c}
	
	\subsubsection*{program1.c}
	
	\lstinputlisting[language=C]{../src/program1.c}
	
	\subsubsection*{program2.c}
	
	\lstinputlisting[language=C]{../src/program2.c}
	
	\subsection*{Выводы}
	
	Вызов $fork$ дублирует породивший его процесс со всеми его переменными, файловыми дескрипторами, приоритетами процесса, рабочий и корневой каталоги, и сегментами выделенной памяти.
	
	Ребёнок {\bf не} наследует:
	\begin{itemize}
		\item идентификатора процесса (PID, PPID);
		\item израсходованного времени ЦП (оно обнуляется);
		\item сигналов процесса-родителя, требующих ответа;
		\item блокированных файлов (record locking).
	\end{itemize}
	
	В процессе выполнения лабораторной работы были приобретены навыки практического применения создания, обработки и отслеживания их состояния. Для выполнения данного варианта задания создание потоков как таковых не требуется, так как всю работу выполняет системный вызов <<exec>>.
	
\end{document}

