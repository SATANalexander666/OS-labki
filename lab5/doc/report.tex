\documentclass[12pt]{article}

\usepackage[utf8x]{inputenc}
\usepackage[T1, T2A]{fontenc}
\usepackage{fullpage}
\usepackage{multicol,multirow}
\usepackage{tabularx}
\usepackage{ulem}
\usepackage{listings} 
\usepackage[english,russian]{babel}
\usepackage{tikz}
\usepackage{pgfplots}
\usepackage{indentfirst}
\usepackage{ulem} 


\parindent=1cm
\makeatletter
\newcommand{\rindex}[2][\imki@jobname]{%
	\index[#1]{\detokenize{#2}}%
}
\makeatother
\newcolumntype{P}[1]{>{\raggedbottom\arraybackslash}p{#1}}

\linespread{1}
\pgfplotsset{compat=1.16}
\begin{document}
	
	\section*{\centering Лабораторная работа №\,5 по курсу :\\ Операционные системы}
	
	Выполнил студент группы М8О-201Б-21 \,\, Кварацхелия Александр.
	
	\subsection*{Условие}
	
	Требуется создать динамические библиотеки, которые реализуют определенный функционал. 
	Далее использовать данные библиотеки 2-мя способами:
	Во время компиляции (на этапе «линковки»/linking)
	Во время исполнения программы. Библиотеки загружаются в память с помощью 
	интерфейса ОС для работы с динамическими библиотеками
	В конечном итоге, в лабораторной работе необходимо получить следующие части:
	Динамические библиотеки, реализующие контракты, которые заданы вариантом;
	Тестовая программа (программа №1), которая используют одну из библиотек, используя 
	знания полученные на этапе компиляции;
	Тестовая программа (программа №2), которая загружает библиотеки, используя только их 
	местоположение и контракты.
	Провести анализ двух типов использования библиотек.
	
	Пользовательский ввод для обоих программ должен быть организован следующим образом:
	Если пользователь вводит команду «0», то программа переключает одну реализацию 
	контрактов на другую (необходимо только для программы №2). Можно реализовать 
	лабораторную работу без данной функции, но максимальная оценка в этом случае будет 
	«хорошо»;
	«1 arg1 arg2 … argN», где после «1» идут аргументы для первой функции, предусмотренной 
	контрактами. После ввода команды происходит вызов первой функции, и на экране 
	появляется результат её выполнения;
	«2 arg1 arg2 … argM», где после «2» идут аргументы для второй функции, 
	предусмотренной контрактами. После ввода команды происходит вызов второй функции, 
	и на экране появляется результат её выполнения.
	
	\subsection*{Задание}
	
	Рассчет интеграла функции $sin(x)$ на отрезке $[A, B]$ с шагом $e$: ($Float SinIntegral$($float$ $A$, $float$ $B$, $float$ $e$)). Методы: Подсчет интеграла методом прямоугольников, подсчет интеграла методом трапеций.
	
	Рассчет производной функции	$float f'(x) = (f(A + deltaX) 	f'(x) = (f(A + deltaX)$
	
	\subsection*{Метод решения}
	
	Использовать утилиты компилятора $gcc$ для динамической "линковки" библиотек.
	
	\subsection*{Код программы}
	
	\subsubsection*{libLab5.h}
	
	\lstinputlisting[language=C]{../include/libLab5.h}
	
	\subsubsection*{lib1.c}
	
	\lstinputlisting[language=C]{../src/lib1.c}
	
	\subsubsection*{lib2.c}
	
	\lstinputlisting[language=C]{../src/lib2.c}
	
	\subsubsection*{program1.c}
	
	\lstinputlisting[language=C]{../src/program1.c}
	
	\subsubsection*{program2.c}
	
	\lstinputlisting[language=C]{../src/program2.c}
	
	\subsection*{Выводы}
	
	В процессе выполнения лабораторной работы были приобретены навыки практического применения создания, обработки и отслеживания состояния динамических библиотек.
	
\end{document}

