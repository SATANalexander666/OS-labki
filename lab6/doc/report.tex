\documentclass[12pt]{article}

\usepackage[utf8x]{inputenc}
\usepackage[T1, T2A]{fontenc}
\usepackage{fullpage}
\usepackage{multicol,multirow}
\usepackage{tabularx}
\usepackage{ulem}
\usepackage{listings} 
\usepackage[english,russian]{babel}
\usepackage{tikz}
\usepackage{pgfplots}
\usepackage{indentfirst}
\usepackage{ulem} 


\parindent=1cm
\makeatletter
\newcommand{\rindex}[2][\imki@jobname]{%
	\index[#1]{\detokenize{#2}}%
}
\makeatother
\newcolumntype{P}[1]{>{\raggedbottom\arraybackslash}p{#1}}

\linespread{1}
\pgfplotsset{compat=1.16}
\begin{document}
	
	\section*{\centering Лабораторная работа №\,6 по курсу :\\ Операционные системы}
	
	Выполнил студент группы М8О-201Б-21 \,\, Кварацхелия Александр.
	
	\subsection*{Условие}
	
	Реализовать распределенную систему по асинхронной обработке запросов. В данной
	распределенной системе должно существовать 2 вида узлов: «управляющий» и
	«вычислительный». Необходимо объединить данные узлы в соответствии с той топологией,
	которая определена вариантом. Связь между узлами необходимо осуществить при помощи
	технологии очередей сообщений. Также в данной системе необходимо предусмотреть проверку
	доступности узлов в соответствии с вариантом. При убийстве («kill -9») любого вычислительного
	узла система должна пытаться максимально сохранять свою работоспособность, а именно все
	дочерние узлы убитого узла могут стать недоступными, но родительские узлы должны сохранить
	свою работоспособность.
	Управляющий узел отвечает за ввод команд от пользователя и отправку этих команд на
	вычислительные узлы.
	
	\subsection*{Задание}
	
	Топология - идеально сбалансированное дерево, команда проверки доступности узла - pingall вывести все недоступные узлы.	
	
	\subsection*{Метод решения}
	
	У клиента есть несколько опций: (регистрируется он в обязательном порядке) создать комнату, присоединиться к комнате, вывести статистику игрока, начать игру. 
	
	\subsection*{Код программы}
	
	\subsubsection*{zmq-utils.hpp}
	
	\lstinputlisting[language=C]{../include/zmq_utils.hpp}
	
	\subsubsection*{server-utils.hpp}
	
	\lstinputlisting[language=C]{../include/server_utils.hpp}
	
	\subsubsection*{zmq-utils.cpp}
	
	\lstinputlisting[language=C]{../src/zmq_utils.cpp}
	
	\subsubsection*{client.cpp}
	
	\lstinputlisting[language=C]{../src/client.cpp}
	
	\subsubsection*{server.cpp}
	
	\lstinputlisting[language=C]{../src/server.cpp}
	
	\subsection*{Выводы}
	
	В процессе выполнения лабораторной работы были приобретены навыки практического применения очередей сообщений, потоков и процессов.
	
\end{document}

